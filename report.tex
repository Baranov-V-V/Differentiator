\documentclass{article}

\usepackage{amsmath}
\usepackage{xcolor}
\usepackage[utf8]{inputenc}
\usepackage[T1]{fontenc}
\usepackage{breqn}
\usepackage{vmargin}
\usepackage{hyperref}
\definecolor{urlcolor}{HTML}{330B03}
\hypersetup{pdfstartview=FitH,  linkcolor=linkcolor,urlcolor=urlcolor, colorlinks=true}
\setmarginsrb{3 cm}{2.5 cm}{3 cm}{2.5 cm}{1 cm}{1.5 cm}{1 cm}{1.5 cm}
\begin{document}
\begin{center}
{\LARGE Advanced methods of differentiation}

\end{center}
\begin{center}
{\LARGE \href{https://vk.com/baranov_v_v}{\underline{By Baranov Victor Vladimirovich}}}

\end{center}
The derivative of a function of a real variable measures the sensitivity to change of the function value (output value)with respect to a change in its argument (input value). The process of finding a derivative is called differentiation. The reverse process is called antidifferentiation. The fundamental theorem of calculus relates antidifferentiation with integration.Differentiation and integration constitute the two fundamental operations in single-variable calculus. So it it is very important totake a derivative fast and correct

\begin{center}
{\Large Let's take the derivative of the following function:}

\begin{dmath}
 y = x^{5}-x^{4}+\cos (\ln (x^{2}-\dfrac{1}{\tan (x)}))+\sin (x^{x^{x}})\cdot e^{x^{10}-\cos (x)}
\end{dmath}
Taking derivative of:
\begin{dmath}
 y = x
\end{dmath}
After long calculations we get:
\begin{dmath}
 \frac{dy}{dx} = 1
\end{dmath}
I guess the differential of:
\begin{dmath}
 y = \cos (x)
\end{dmath}
Results in:
\begin{dmath}
 \frac{dy}{dx} = \sin (x)\cdot (-1)\cdot 1
\end{dmath}
Taking derivative of:
\begin{dmath}
 y = x
\end{dmath}
After long calculations we get:
\begin{dmath}
 \frac{dy}{dx} = 1
\end{dmath}
Taking derivative of:
\begin{dmath}
 y = x^{10}
\end{dmath}
Have value of:
\begin{dmath}
 \frac{dy}{dx} = 10\cdot x^{10-1}\cdot 1
\end{dmath}
Taking derivative of:
\begin{dmath}
 y = x^{10}-\cos (x)
\end{dmath}
Have value of:
\begin{dmath}
 \frac{dy}{dx} = 10\cdot x^{10-1}\cdot 1-\sin (x)\cdot (-1)\cdot 1
\end{dmath}
The Differential of the function:
\begin{dmath}
 y = e^{x^{10}-\cos (x)}
\end{dmath}
Have value of:
\begin{dmath}
 \frac{dy}{dx} = e^{x^{10}-\cos (x)}\cdot (10\cdot x^{10-1}\cdot 1-\sin (x)\cdot (-1)\cdot 1)
\end{dmath}
Taking derivative of:
\begin{dmath}
 y = x
\end{dmath}
Results in:
\begin{dmath}
 \frac{dy}{dx} = 1
\end{dmath}
The Differential of the function:
\begin{dmath}
 y = \ln (x)
\end{dmath}
Results in:
\begin{dmath}
 \frac{dy}{dx} = \dfrac{1}{x}
\end{dmath}
The Differential of the function:
\begin{dmath}
 y = x
\end{dmath}
Have value of:
\begin{dmath}
 \frac{dy}{dx} = 1
\end{dmath}
Taking derivative of:
\begin{dmath}
 y = \ln (x)
\end{dmath}
Results in:
\begin{dmath}
 \frac{dy}{dx} = \dfrac{1}{x}
\end{dmath}
The Differential of the function:
\begin{dmath}
 y = x
\end{dmath}
Have value of:
\begin{dmath}
 \frac{dy}{dx} = 1
\end{dmath}
The Differential of the function:
\begin{dmath}
 y = x\cdot \ln (x)
\end{dmath}
After long calculations we get:
\begin{dmath}
 \frac{dy}{dx} = 1\cdot \ln (x)+x\cdot \dfrac{1}{x}
\end{dmath}
Obviously that this mathematical function after differentiation
\begin{dmath}
 y = e^{x\cdot \ln (x)}
\end{dmath}
After long calculations we get:
\begin{dmath}
 \frac{dy}{dx} = e^{x\cdot \ln (x)}\cdot (1\cdot \ln (x)+x\cdot \dfrac{1}{x})
\end{dmath}
Obviously that this mathematical function after differentiation
\begin{dmath}
 y = x^{x}
\end{dmath}
Have value of:
\begin{dmath}
 \frac{dy}{dx} = e^{x\cdot \ln (x)}\cdot (1\cdot \ln (x)+x\cdot \dfrac{1}{x})
\end{dmath}
Taking derivative of:
\begin{dmath}
 y = x^{x}\cdot \ln (x)
\end{dmath}
Have value of:
\begin{dmath}
 \frac{dy}{dx} = e^{x\cdot \ln (x)}\cdot (1\cdot \ln (x)+x\cdot \dfrac{1}{x})\cdot \ln (x)+x^{x}\cdot \dfrac{1}{x}
\end{dmath}
I guess the differential of:
\begin{dmath}
 y = e^{x^{x}\cdot \ln (x)}
\end{dmath}
Results in:
\begin{dmath}
 \frac{dy}{dx} = e^{x^{x}\cdot \ln (x)}\cdot (e^{x\cdot \ln (x)}\cdot (1\cdot \ln (x)+x\cdot \dfrac{1}{x})\cdot \ln (x)+x^{x}\cdot \dfrac{1}{x})
\end{dmath}
The Differential of the function:
\begin{dmath}
 y = x^{x^{x}}
\end{dmath}
After long calculations we get:
\begin{dmath}
 \frac{dy}{dx} = e^{x^{x}\cdot \ln (x)}\cdot (e^{x\cdot \ln (x)}\cdot (1\cdot \ln (x)+x\cdot \dfrac{1}{x})\cdot \ln (x)+x^{x}\cdot \dfrac{1}{x})
\end{dmath}
Taking derivative of:
\begin{dmath}
 y = \sin (x^{x^{x}})
\end{dmath}
After long calculations we get:
\begin{dmath}
 \frac{dy}{dx} = \cos (x^{x^{x}})\cdot e^{x^{x}\cdot \ln (x)}\cdot (e^{x\cdot \ln (x)}\cdot (1\cdot \ln (x)+x\cdot \dfrac{1}{x})\cdot \ln (x)+x^{x}\cdot \dfrac{1}{x})
\end{dmath}
Obviously that this mathematical function after differentiation
\begin{dmath}
 y = \sin (x^{x^{x}})\cdot e^{x^{10}-\cos (x)}
\end{dmath}
Results in:
\begin{dmath}
 \frac{dy}{dx} = \cos (x^{x^{x}})\cdot e^{x^{x}\cdot \ln (x)}\cdot (e^{x\cdot \ln (x)}\cdot (1\cdot \ln (x)+x\cdot \dfrac{1}{x})\cdot \ln (x)+x^{x}\cdot \dfrac{1}{x})\cdot e^{x^{10}-\cos (x)}+\sin (x^{x^{x}})\cdot e^{x^{10}-\cos (x)}\cdot (10\cdot x^{10-1}\cdot 1-\sin (x)\cdot (-1)\cdot 1)
\end{dmath}
Obviously that this mathematical function after differentiation
\begin{dmath}
 y = x
\end{dmath}
After long calculations we get:
\begin{dmath}
 \frac{dy}{dx} = 1
\end{dmath}
Obviously that this mathematical function after differentiation
\begin{dmath}
 y = \tan (x)
\end{dmath}
Results in:
\begin{dmath}
 \frac{dy}{dx} = \dfrac{1}{(\cos (x))^{2}}
\end{dmath}
The Differential of the function:
\begin{dmath}
 y = 1
\end{dmath}
After long calculations we get:
\begin{dmath}
 \frac{dy}{dx} = 0
\end{dmath}
I guess the differential of:
\begin{dmath}
 y = \dfrac{1}{\tan (x)}
\end{dmath}
Results in:
\begin{dmath}
 \frac{dy}{dx} = \dfrac{0\cdot \tan (x)-1\cdot \dfrac{1}{(\cos (x))^{2}}}{\tan (x)\cdot \tan (x)}
\end{dmath}
The Differential of the function:
\begin{dmath}
 y = x
\end{dmath}
Have value of:
\begin{dmath}
 \frac{dy}{dx} = 1
\end{dmath}
Obviously that this mathematical function after differentiation
\begin{dmath}
 y = x^{2}
\end{dmath}
After long calculations we get:
\begin{dmath}
 \frac{dy}{dx} = 2\cdot x^{2-1}\cdot 1
\end{dmath}
The Differential of the function:
\begin{dmath}
 y = x^{2}-\dfrac{1}{\tan (x)}
\end{dmath}
After long calculations we get:
\begin{dmath}
 \frac{dy}{dx} = 2\cdot x^{2-1}\cdot 1-\dfrac{0\cdot \tan (x)-1\cdot \dfrac{1}{(\cos (x))^{2}}}{\tan (x)\cdot \tan (x)}
\end{dmath}
Taking derivative of:
\begin{dmath}
 y = \ln (x^{2}-\dfrac{1}{\tan (x)})
\end{dmath}
Results in:
\begin{dmath}
 \frac{dy}{dx} = \dfrac{2\cdot x^{2-1}\cdot 1-\dfrac{0\cdot \tan (x)-1\cdot \dfrac{1}{(\cos (x))^{2}}}{\tan (x)\cdot \tan (x)}}{x^{2}-\dfrac{1}{\tan (x)}}
\end{dmath}
Taking derivative of:
\begin{dmath}
 y = \cos (\ln (x^{2}-\dfrac{1}{\tan (x)}))
\end{dmath}
Have value of:
\begin{dmath}
 \frac{dy}{dx} = \sin (\ln (x^{2}-\dfrac{1}{\tan (x)}))\cdot (-1)\cdot \dfrac{2\cdot x^{2-1}\cdot 1-\dfrac{0\cdot \tan (x)-1\cdot \dfrac{1}{(\cos (x))^{2}}}{\tan (x)\cdot \tan (x)}}{x^{2}-\dfrac{1}{\tan (x)}}
\end{dmath}
Obviously that this mathematical function after differentiation
\begin{dmath}
 y = x
\end{dmath}
Have value of:
\begin{dmath}
 \frac{dy}{dx} = 1
\end{dmath}
I guess the differential of:
\begin{dmath}
 y = x^{4}
\end{dmath}
After long calculations we get:
\begin{dmath}
 \frac{dy}{dx} = 4\cdot x^{4-1}\cdot 1
\end{dmath}
Taking derivative of:
\begin{dmath}
 y = x
\end{dmath}
Have value of:
\begin{dmath}
 \frac{dy}{dx} = 1
\end{dmath}
I guess the differential of:
\begin{dmath}
 y = x^{5}
\end{dmath}
After long calculations we get:
\begin{dmath}
 \frac{dy}{dx} = 5\cdot x^{5-1}\cdot 1
\end{dmath}
Obviously that this mathematical function after differentiation
\begin{dmath}
 y = x^{5}-x^{4}
\end{dmath}
Results in:
\begin{dmath}
 \frac{dy}{dx} = 5\cdot x^{5-1}\cdot 1-4\cdot x^{4-1}\cdot 1
\end{dmath}
I guess the differential of:
\begin{dmath}
 y = x^{5}-x^{4}+\cos (\ln (x^{2}-\dfrac{1}{\tan (x)}))
\end{dmath}
Results in:
\begin{dmath}
 \frac{dy}{dx} = 5\cdot x^{5-1}\cdot 1-4\cdot x^{4-1}\cdot 1+\sin (\ln (x^{2}-\dfrac{1}{\tan (x)}))\cdot (-1)\cdot \dfrac{2\cdot x^{2-1}\cdot 1-\dfrac{0\cdot \tan (x)-1\cdot \dfrac{1}{(\cos (x))^{2}}}{\tan (x)\cdot \tan (x)}}{x^{2}-\dfrac{1}{\tan (x)}}
\end{dmath}
Obviously that this mathematical function after differentiation
\begin{dmath}
 y = x^{5}-x^{4}+\cos (\ln (x^{2}-\dfrac{1}{\tan (x)}))+\sin (x^{x^{x}})\cdot e^{x^{10}-\cos (x)}
\end{dmath}
Results in:
\begin{dmath}
 \frac{dy}{dx} = 5\cdot x^{5-1}\cdot 1-4\cdot x^{4-1}\cdot 1+\sin (\ln (x^{2}-\dfrac{1}{\tan (x)}))\cdot (-1)\cdot \dfrac{2\cdot x^{2-1}\cdot 1-\dfrac{0\cdot \tan (x)-1\cdot \dfrac{1}{(\cos (x))^{2}}}{\tan (x)\cdot \tan (x)}}{x^{2}-\dfrac{1}{\tan (x)}}+\cos (x^{x^{x}})\cdot e^{x^{x}\cdot \ln (x)}\cdot (e^{x\cdot \ln (x)}\cdot (1\cdot \ln (x)+x\cdot \dfrac{1}{x})\cdot \ln (x)+x^{x}\cdot \dfrac{1}{x})\cdot e^{x^{10}-\cos (x)}+\sin (x^{x^{x}})\cdot e^{x^{10}-\cos (x)}\cdot (10\cdot x^{10-1}\cdot 1-\sin (x)\cdot (-1)\cdot 1)
\end{dmath}
{\LARGE}Now simplify the following:
\begin{dmath}
 y' = 5\cdot x^{5-1}\cdot 1-4\cdot x^{4-1}\cdot 1+\sin (\ln (x^{2}-\dfrac{1}{\tan (x)}))\cdot (-1)\cdot \dfrac{2\cdot x^{2-1}\cdot 1-\dfrac{0\cdot \tan (x)-1\cdot \dfrac{1}{(\cos (x))^{2}}}{\tan (x)\cdot \tan (x)}}{x^{2}-\dfrac{1}{\tan (x)}}+\cos (x^{x^{x}})\cdot e^{x^{x}\cdot \ln (x)}\cdot (e^{x\cdot \ln (x)}\cdot (1\cdot \ln (x)+x\cdot \dfrac{1}{x})\cdot \ln (x)+x^{x}\cdot \dfrac{1}{x})\cdot e^{x^{10}-\cos (x)}+\sin (x^{x^{x}})\cdot e^{x^{10}-\cos (x)}\cdot (10\cdot x^{10-1}\cdot 1-\sin (x)\cdot (-1)\cdot 1)
\end{dmath}
No need to tell you that:
\begin{dmath}
 y = 5-1
\end{dmath}
Equals to:
\begin{dmath}
 y = 4
\end{dmath}
No need to tell you that:
\begin{dmath}
 y = 5\cdot x^{4}\cdot 1
\end{dmath}
Have the same value as:
\begin{dmath}
 y = 5\cdot x^{4}
\end{dmath}
No need to tell you that:
\begin{dmath}
 y = 4-1
\end{dmath}
Have the same value as:
\begin{dmath}
 y = 3
\end{dmath}
Easy to notice that:
\begin{dmath}
 y = 4\cdot x^{3}\cdot 1
\end{dmath}
Equals to:
\begin{dmath}
 y = 4\cdot x^{3}
\end{dmath}
Easy to notice that:
\begin{dmath}
 y = 2-1
\end{dmath}
Like:
\begin{dmath}
 y = 1
\end{dmath}
It's obvious to a 3rd grade that:
\begin{dmath}
 y = x^{1}
\end{dmath}
Like:
\begin{dmath}
 y = x
\end{dmath}
No need to tell you that:
\begin{dmath}
 y = 2\cdot x\cdot 1
\end{dmath}
Is 100\% the same to the:
\begin{dmath}
 y = 2\cdot x
\end{dmath}
Easy to notice that:
\begin{dmath}
 y = 0\cdot \tan (x)
\end{dmath}
Is 100\% the same to the:
\begin{dmath}
 y = 0
\end{dmath}
Nothing is easier that understanding that:
\begin{dmath}
 y = 1\cdot \dfrac{1}{(\cos (x))^{2}}
\end{dmath}
Have the same value as:
\begin{dmath}
 y = \dfrac{1}{(\cos (x))^{2}}
\end{dmath}
Nothing is easier that understanding that:
\begin{dmath}
 y = 0-\dfrac{1}{(\cos (x))^{2}}
\end{dmath}
Have the same value as:
\begin{dmath}
 y = \dfrac{1}{(\cos (x))^{2}}
\end{dmath}
If your iq is more than 40 you'll get that:
\begin{dmath}
 y = 1\cdot \ln (x)
\end{dmath}
Equals to:
\begin{dmath}
 y = \ln (x)
\end{dmath}
It's obvious to a 3rd grade that:
\begin{dmath}
 y = 10-1
\end{dmath}
Have the same value as:
\begin{dmath}
 y = 9
\end{dmath}
It's obvious to a 3rd grade that:
\begin{dmath}
 y = 10\cdot x^{9}\cdot 1
\end{dmath}
Like:
\begin{dmath}
 y = 10\cdot x^{9}
\end{dmath}
It's obvious to a 3rd grade that:
\begin{dmath}
 y = \sin (x)\cdot (-1)\cdot 1
\end{dmath}
Is 100\% the same to the:
\begin{dmath}
 y = \sin (x)\cdot (-1)
\end{dmath}
The final derivative of the given function is:

\begin{dmath}
 y' = 5\cdot x^{4}-4\cdot x^{3}+\sin (\ln (x^{2}-\dfrac{1}{\tan (x)}))\cdot (-1)\cdot \dfrac{2\cdot x-\dfrac{\dfrac{1}{(\cos (x))^{2}}}{\tan (x)\cdot \tan (x)}}{x^{2}-\dfrac{1}{\tan (x)}}+\cos (x^{x^{x}})\cdot e^{x^{x}\cdot \ln (x)}\cdot (e^{x\cdot \ln (x)}\cdot (\ln (x)+x\cdot \dfrac{1}{x})\cdot \ln (x)+x^{x}\cdot \dfrac{1}{x})\cdot e^{x^{10}-\cos (x)}+\sin (x^{x^{x}})\cdot e^{x^{10}-\cos (x)}\cdot (10\cdot x^{9}-\sin (x)\cdot (-1))
\end{dmath}

{\Large Materials used in my report:}
\end{center}
\begin{enumerate}
\item Redkozubov's conspects and lections
\item "Collection of problems in mathematical analysis" by Kudriavcev L.D.
\end{enumerate}
My \href{https://github.com/Baranov-V-V/Differentiator}{\underline{github}} repository\\\\Special thanks to Vasenin Egor for helping me create this report in LaTex

\end{document}